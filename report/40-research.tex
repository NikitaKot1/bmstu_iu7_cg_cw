\chapter{Исследовательская часть}

В данном разделе будут приведены примеры работы программ, постановка эксперимента и сравнительный анализ алгоритмов на основе полученных данных.

\section{Технические характеристики}

Технические характеристики устройства, на котором выполнялось тестирование:

\begin{itemize}
	\item операционная система Manjaro xfce \cite{ubuntu} Linux \cite{linux} x86\_64;
	\item память 8 ГБ;
	\item мобильный процессор AMD Ryzen™ 7 3700U @ 2.3 ГГц \cite{intel}.
\end{itemize}

Тестирование проводилось на ноутбуке, включенном в сеть электропитания. Во время тестирования ноутбук был нагружен только встроенными приложениями окружения, а также непосредственно системой тестирования.

\section{Влияние площади, занимаемой моделью на эране, на время отрисовки}

В алгоритме, который использует Z-буфер, происходит попиксельный анализ. В таблице \ref{tab:time1} продемонстрировано пользовательское время программы при разной площади модели.

\begin{table}[ht!]
	\begin{center}
		\caption{Пользовательское время работы программы при разной площади, занимаемой моделью}
		\label{tab:time1}
		\begin{tabular}{|c|c|}
			\hline
			Площадь модели & Время в с. \\
			\hline
			$50 \times 50$  &  945657 \\
			\hline
			$100 \times 100$  & 2044827 \\
			\hline
			$150 \times 150$  & 4729681 \\
			\hline
			$200 \times 200$ & 6306256 \\
			\hline
			$250 \times 250$ & 11384766 \\
			\hline
			$300 \times 300$ & 17261890 \\
			\hline
			$350 \times 350$ & 23556005 \\
			\hline
			$400 \times 400$ & 30835890 \\
			\hline
		\end{tabular}
	\end{center}
\end{table}

\FloatBarrier

На рисунке \ref{graph:r} представлено время работы алгоритма, использующего z-буфер.

\begin{figure}[ht!]
	\captionsetup{singlelinecheck = false, justification=centering}
	\centering
	\begin{tikzpicture}
		\begin{axis}[
			xlabel={линейный размер площади},
			ylabel={время, наносекунды},
			width=0.95\textwidth,
			height=0.3\textheight,
			xmin=0, xmax=450,
			legend pos=north west,
			xmajorgrids=true,
			grid style=dashed,
			]
			\addplot[
			color=blue,
			mark=asterisk
			]
			table [x=N, y=time]{
				N time
				50 945657
				100 2044827
				150 4729681
				200 6306256
				250 11384766
				300 17261890
				350 23556005
				400 30835890
			};
		\end{axis}
		
	\end{tikzpicture}
	\caption{Время работы алгоритма, использующего z-буфер}
	\label{graph:r}
\end{figure}

\FloatBarrier

\section*{Вывод}

В результате было выявлено, что время работы реализации алгоритма растет линейно относительно площади на экране отрисовки, занимаемой моделью --- квадратично относительно сторон площади модели.