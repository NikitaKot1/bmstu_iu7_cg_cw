\chapter*{Введение}
\addcontentsline{toc}{chapter}{Введение}

Компьютерная графика (также машинная графика) — область деятельности, в которой компьютеры наряду со специальным программным обеспечением используются в качестве инструмента как для создания (синтеза) и редактирования изображений, так и для оцифровки визуальной информации, полученной из реального мира, с целью дальнейшей ее обработки и хранения.

В современном мире яркими примерами использования компьютерной графики являются: визуализация данных на производстве, создание реалистичных специальных эффектов в кино и для создания видеоигр.
При этом из-за разницы в задачах могут быть использованы различные подходы и алгоритмы. Так, компьютерная графика в кино требует реалистичного изображения, в то время как задача визуализации данных в производстве требует лишь понятного представления трехмерного объекта.

B компьютерной графике существует множество разнообразных алгоритмов, помогающих решить эти задачи, однако большинство являются достаточно ресурсоемкими, так как чем более реалистичное изображение мы хотим получить, тем больше нам необходимо времени и памяти на синтез. При этом, если от изображения требуется быть интуитивно понятным представлением трехмерного объекта, также не стоит пренебрегать долей реалистичности.

Целью данной курсовой работы является приложение для создания и редактирования трехмерных объектов, которые впоследствии можно будет использовать в различных нуждах.

Чтобы достигнуть поставленной цели, требуется решить следующие задачи:
\begin{itemize}
	\item[-] проанализировать существующие представления трехмерных объектов и обосновать выбор тех из них, которые в наибольшей степени подходят для решения поставленной задачи;
	\item[-] проанализировать существующие алгоритмы построения изображения и обосновать выбор тех из них, которые в наибольшей степени подходят для решения поставленной задачи;
	\item[-] реализовать выбранные алгоритмы;
	\item[-] реализовать возможность изменения положения модели в пространстве;
	\item[-] реализовать возможность создания полигонов как части модели;
	\item[-] реализовать возможность удаления полигонов модели;
	\item[-] реализовать возможность изменения формы модели;
	\item[-] реализовать возможность сохранения модели в файл;
	\item[-] реализовать возможность чтения модели из файла;
	\item[-] реализовать возможность изменение цвета грани модели;
\end{itemize}