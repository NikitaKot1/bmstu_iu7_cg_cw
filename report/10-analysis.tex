\chapter{Аналитический раздел}

В данной части проводится анализ объектов сцены и существующих алгоритмов построения изображений и выбор более подходящих алгоритмов для дальнейшего использования.

\section{Описание и формализация объектов сцены}

Источник света – материальная точка в пространстве, из которой исходят лучи во все стороны. Положение на сцене регулируется координатами (x, y, z), может иметь цвет.

Для описания трехмерных геометрических объектов существует три модели: каркасная, поверхностная и объемная. Для реализации поставленной задачи, более подходящей моделью будет поверхностная, так как, по сравнению с каркасной, она может дать более реалистичное и понятное зрителю изображение. В то же время, объёмной модели требуется больше памяти и поэтому она будет скорее излишняя, в условиях моей задачи.

B свою очередь поверхностная модель может задаваться параметрическим представлением или полигональной сеткой.

При параметрическом представлении поверхность можно получить при вычислении параметрической функции, что очень удобно при просчете поверхностей вращения, однако, ввиду их отсутствия, данное представление будет не выгодно.

В случае полигональной сетки форма объекта задаётся некоторой совокупностью вершин, рёбер и граней, что позволяет выделить несколько способов представления:

\begin{enumerate}[label=\arabic*)]
	\item Вершинное представление --- при таком представлении вершины хранят указатели на соседние вершины. В таком случае для рендеринга нужно будет обойти все данные по списку, что может занимать достаточно много времени при переборе.
	\item Список граней --- при таком представлении объект хранится, как множество граней и вершин. В таком случае достаточно удобно производить различные манипуляции над данными.
	\item Таблица углов --- при таком представлении вершины хранятся в предопределенной таблице, такой что обход таблицы неявно задает полигоны. Такое представление более компактное и более производительное для нахождения полигонов, однако операции по замене достаточно медлительны.
\end{enumerate}

Наиболее подходящим представлением сцены в условиях поставленной задачи будет представление в виде списка граней, т.к. оно позволяет эффективно манипулировать данными, а также позволяет проводить явный поиск вершин грани и самих граней, которые окружают вершину.

\section{Анализ алгоритмов удаления невидимых линий и поверхностей}

Выбирая подходящий алгоритм удаления невидимых линий, необходимо учитывать поставленную задачу и ее особенности, а именно то, что сцена должна быть динамической, а следовательно, алгоритм должен быть достаточно быстрым.В настоящее время существует множество различных алгоритмов построения изображений: алгоритм трассировки лучей, алгоритм Варнока, алгоритм Робертса, алгоритм Z-буфера и т.д. Рассмотрим преимущества и недостатки каждого из алгоритмов.

\subsection{Алгоритм трассировки лучей}

Данный алгоритм обладает следующими преимуществами:
\begin{itemize}
	\item[-] вычислительная сложность слабо зависит от сложности сцены;
	\item[-] отсечение невидимых поверхностей, перспектива и корректное изменения поля зрения являются логическим следствием алгоритма;
	\item[-] возможность рендеринга гладких объектов без аппроксимации их полигональными поверхностями;
	\item[-] возможность параллельно и независимо трассировать два и более лучей, разделять участки (или зоны экрана) для трассирования на разных узлах кластера и т.д.
\end{itemize}

Однако, недостатком алгоритма является его производительность. Метод трассирования лучей каждый раз начинает процесс определения цвета пикселя заново, рассматривая каждый луч наблюдения в отдельности \cite{raytrace}.

В поставленной мною задаче не используются явления отражения и преломления света и поэтому некоторые вычисления окажутся излишними, кроме того, скорость синтеза сцены должна быть высокой, из-за чего алгоритм трассировки лучей не подходит.

\subsection{Алгоритм Варнока}

Преимуществом данного алгоритма является то, что он работает в пространстве изображений. Алгоритм предлагает разбиение области рисунка на более мелкие окна, и для каждого такого окна определяются связанные с ней многоугольники и те, видимость которых «легко» определить, изображаются на сцене \cite{varnok}.

К недостаткам можно отнести то, что в противном случае разбиение повторяется, и для каждой из вновь полученных подобластей рекурсивно применяется процедура принятия решения. По предположению, с уменьшением размеров области, её будет перекрывать всё меньшее количество многоугольников. Считается, что в пределе будут получены области, которые содержат не более одного многоугольника, и решение будет принято достаточно просто.

Из-за того, что поиск может продолжаться до тех пор, пока либо остаются области, содержащие не один многоугольник, либо пока размер области не станет совпадать с одним пикселем, алгоритм не подходит под условия моей задачи.

\subsection{Алгоритм Робертса}

Данный алгоритм обладает достаточно простыми и одновременно мощными математическими методами, что, несомненно, является преимуществом. Некоторые реализации, например использующие предварительную сортировку по оси z, могут похвастаться линейной зависимостью от числа объектов на сцене \cite{roberts}.

Однако алгоритм обладает недостатком, который заключается в вычислительной трудоемкости, растущей теоретически, как квадрат числа объектов, что может негативно сказаться на производительности. В то же время реализация оптимизированных алгоритмов достаточно сложна.

\subsection{Алгоритм, использующий Z буфер}

Главными преимуществами данного алгоритма являются его достаточная простота реализации и функциональность, которая более чем подходит для визуализации динамической сцены. Также алгоритм не тратит время на сортировку элементов сцены, что даёт значительный прирост к производительности \cite{zbufer}.

К недостатку алгоритма можно отнести большой объём памяти необходимый для хранения информации о каждом пикселе. Однако данный недостаток является незначительным ввиду того, что большинство современных компьютеров обладает достаточно большим объёмом памяти для корректной работы алгоритма.

\subsection*{Вывод}

Алгоритм, использующий Z-буфер, является наиболее подходящим для создания и редактирования трехмерных моделей. На его основе будет просто визуализировать схематичную модель, показывающую требуемые в производстве нюансы.


\section{Анализ моделей освещения}

На сегодняшний день существует две модели освещения, которые используются для построения света на трехмерных сценах: локальная и глобальная. Исходя из поставленной задачи, необходимо выбрать более производительный вариант, который позволит достаточно быстро просчитывать освещения на сцене.

\subsection{Локальная модель освещения}

Является самой простой и не рассматривает процессы светового взаимодействия объектов сцены между собой и рассчитывает освещённость только самих объектов \cite{lamber_fong}.

\subsection{Глобальная модель освещения}

Данная модель рассматривает трехмерную сцену, как единую систему и описывает освещение с учетом взаимного влияния объектов, что позволяет рассматривать такие явления, как многократное отражение и преломление света, а также рассеянное освещение \cite{lamber_fong}.

\subsection*{Вывод}

Так как программа не должна выводить реалистичного изображения, рациональнее будет использовать локальную модель освещения, что значительно упростит вычисления и повысит производительность редактора.


\section{Анализ алгоритмов закрашивания}

\subsection{Алгоритм простой закраски}

По закону Ламберта, вся грань закрашивается одним уровнем интенсивности. Метод является достаточно простым в реализации и не требовательным к ресурсам. Однако алгоритм плохо учитывает отражения и при отрисовки тел вращения, возникают проблемы \cite{coloring}.

\subsection{Алгоритм закраски по Гуро}

В основу данного алгоритма положена билинейная интерполяция интенсивностей, позволяющая устранять дискретность изменения интенсивности. Благодаря этому криволинейные поверхности будут более гладкими\cite{coloring}.


\subsection{Алгоритм закраски по Фонгу}

В данном алгоритме за основу берётся билинейная интерполяция векторов нормалей, благодаря чему достигается лучшая локальная аппроксимация кривизны поверхности и, следовательно, изображение выглядит более реалистичным\cite{coloring}.

Однако алгоритм требует больших вычислений, по сравнению с алгоритмом по Гуро, так как происходит интерполяция значений векторов нормалей.

\subsection*{Вывод}

В условиях поставленной задачи, рациональнее будет использовать алгоритм простой закраски, так как программа не должна выводить реалистичное изображение и при этом должна быть достаточно быстрой. Кроме того, алгоритм достаточно хорошо сочетается с выбранным ранее Z буфером.